\section{Reflection and Retrospective}

\subsection{Team Retrospective}
Nhóm đã tổ chức một buổi retrospective sau khi hoàn thành dự án, sử dụng khung Start--Stop--Continue để đánh giá quá trình làm việc.

\begin{table}[H]
\centering
\begin{tabular}{|c|p{10cm}|}
\hline
\textbf{Category} & \textbf{Reflection} \\
\hline
Start & Bổ sung kiểm thử tự động và viết tài liệu kỹ thuật sớm hơn trong sprint. \\
\hline
Stop & Tránh dồn các công việc tích hợp vào giai đoạn cuối sprint. \\
\hline
Continue & Tiếp tục sử dụng phân chia layer và service rõ ràng để dễ bảo trì. \\
\hline
\end{tabular}
\caption{Start--Stop--Continue retrospective}
\end{table}

\subsection{Lessons Learned}
Thông qua dự án SERS, nhóm học được cách chuyển đổi yêu cầu nghiệp vụ thành kiến trúc phần mềm thực tế, triển khai scheduler và xử lý các trường hợp phức tạp như sự kiện lặp. Việc áp dụng Scrum giúp nhóm kiểm soát tiến độ và giảm rủi ro trong phát triển.

\subsection{Future Improvements}
Nếu tiếp tục phát triển, hệ thống có thể được mở rộng với các tính năng như xác thực người dùng, push notification nền, message queue cho scheduler, và chuyển sang hệ quản trị cơ sở dữ liệu phù hợp hơn cho môi trường production.
