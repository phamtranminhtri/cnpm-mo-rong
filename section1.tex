\section{Requirements Engineering}

\subsection{Stakeholders}

Đối với hệ thống  Smart Event Reminder System (SERS), các stakeholder bao gồm:

\begin{itemize}
    \item User (người dùng): Người sử dụng SERS, dùng vào việc tạo và quản lý các sự kiện
    \item Admin (quản trị viên): Người quản lý phần mềm hệ thống, các tài khoản user, và hệ cơ sở dữ liệu của hệ thống.
    \item Project team (nhóm dự án): Chịu trách nhiệm thiết kế và phát triển hệ thống SERS.
    \item Notification service (dịch vụ thông báo): Có nhiệm vụ gởi thông báo về các sự kiện đến user. 
\end{itemize}

\subsection{Functional Requirements \& Non-Functional 
Requirements}

\begin{enumerate}
    \item Functional requirements:
    \begin{itemize}
        \item \textbf{FR-01:} Người dùng có thể tạo một sự kiện, với tiêu đề, mô tả, ngày giờ, địa điểm, tần suất lặp lại (mỗi ngày, mỗi tuần, mỗi tháng).
        \item \textbf{FR-02:} Người dùng có thể thêm hoặc cập nhật các chi tiết của một sự kiện.
        \item \textbf{FR-03:} Người dùng có thể xóa sự kiện bất kỳ.
        \item \textbf{FR-04:} Người dùng có thể xem danh sách các sự kiện sắp tới.
        \item \textbf{FR-05:} Người dùng sẽ nhận được thông báo nhắc nhở từ hệ thống trước khi sự kiện diễn ra (15 phút). 
        \item \textbf{FR-06:} Cho phép người dùng tạo liên kết URL để chia sẻ sự kiện cho người dùng khác xem.
    \end{itemize}

    \item Non-functional requirements:
    \begin{itemize}
        \item \textbf{NFR-01:} Giao diện cần dễ hiểu, dễ sử dụng để cho một người dùng mới có thể nhanh chóng đăng nhập và tạo sự kiện mà không cần đọc hướng dẫn.
        \item \textbf{NFR-02:} Hệ thống cần hiển thị danh sách các sự kiện trong vòng 1 giây khi người dùng có dưới 100 sự kiện trong database.
        \item \textbf{NFR-03:} Dịch vụ thông báo cần gửi nhắc nhở đến người dùng 15 phút trước khi sự kiện bắt đầu, với tỉ lệ gửi thông báo thành công 99\%.
    \end{itemize}
\end{enumerate}

\subsection{User stories}

\begin{itemize}
    \item \textbf{US-01:} \emph{Là một} sinh viên, \emph{tôi muốn} \textbf{tạo} và \textbf{xem} danh sách hạn nộp bài tập sắp tới  \emph{để} sắp xếp thời gian học tập tốt hơn.
    \item \textbf{US-02:} \emph{Là một} gia sư, \emph{tôi muốn} \textbf{nhận thông báo} kịp thời về các buổi dạy kèm \emph{để} không quên lịch dạy của tôi.
    \item \textbf{US-04:} \emph{Là một} bệnh nhân, \emph{tôi muốn} có nhắc nhở \textbf{lặp lại} hằng tuần \emph{để} giúp tôi nhớ loại thuốc cần uống trong ngày.
    \item \textbf{US-03:} \emph{Là một} thành viên câu lạc bộ, \emph{tôi muốn} \textbf{chia sẻ} sự kiện cá nhân cho bạn bè \emph{để} mọi người có thể biết đến thời gian và địa điểm của sự kiện và tham gia.
    \item \textbf{US-05:} \emph{Là một} nhân viên văn phòng, \emph{tôi muốn} {đồng bộ} dữ liệu về sự kiện của mình trong hệ thống lên Google Calendar \emph{để} xem lịch tiện lợi hơn
\end{itemize}


\subsection{Acceptance criteria \& MoSCoW}

Dựa theo phương pháp MoSCoW \cite{gfg_moscow_2025}, ta xếp loại độ ưu tiên các user story thành 1 trong 4 nhóm: Must have, Should have, Could have, Won't have.

\begin{table}[h!]
\centering
\begin{tabularx}{\textwidth}{|p{1.2cm}|p{3.2cm}|p{2.2cm}|X|}
\hline
\textbf{ID} & \textbf{User Story} & \textbf{MoSCoW} & \textbf{Acceptance Criteria} \\ \hline

US-01 & Tạo và xem danh sách bài tập (Sinh viên) & Must Have &
(1) Hệ thống hiển thị form nhập liệu với các trường: Tiêu đề, Mô tả, Ngày giờ, Địa điểm.  
(2) Nút ``Lưu'' bị vô hiệu hóa hoặc báo lỗi nếu bỏ trống trường ``Tiêu đề'' hoặc ``Ngày giờ''.  
(3) Sau khi lưu thành công, sự kiện mới phải xuất hiện ngay lập tức trong danh sách ``Sắp tới''.  
(4) Ngày giờ của sự kiện phải là một thời điểm trong tương lai. \\ \hline

US-02 & Nhận thông báo dạy kèm (Gia sư) & Must Have &
(1) Hệ thống gửi thông báo đúng 15 phút trước giờ sự kiện.  
(2) Nội dung thông báo hiển thị đúng tiêu đề của sự kiện.  
(3) Nếu người dùng đang offline, thông báo hiển thị khi họ đăng nhập lại. \\ \hline

US-04 & Nhắc nhở lặp lại uống thuốc (Bệnh nhân) & Should Have &
(1) Trong form tạo sự kiện có tùy chọn ``Lặp lại'': Hàng ngày, Hàng tuần.  
(2) Nếu chọn ``Hàng tuần'', hệ thống tự động tạo các bản sao sự kiện vào cùng giờ các ngày tiếp theo.  
(3) Khi xóa một sự kiện lặp lại, hệ thống hỏi người dùng muốn xóa ``Chỉ sự kiện này'' hay ``Tất cả chuỗi''. \\ \hline

US-03 & Chia sẻ sự kiện (Thành viên CLB) & Could Have &
(1) Màn hình chi tiết sự kiện có nút ``Chia sẻ'' hoặc ``Get Link''.  
(2) Hệ thống sinh ra URL duy nhất (ví dụ: sers.com/share/xyz123).  
(3) Khi truy cập URL không cần đăng nhập, người xem thấy Tiêu đề, Thời gian, Địa điểm (Read-only). \\ \hline

US-05 & Đồng bộ Google Calendar (Nhân viên VP) & Won't Have &
(1) Người dùng thấy nút ``Sync to Google''.  
(2) Hệ thống yêu cầu quyền truy cập (Mockup OAuth).  
(3) Sự kiện được gửi thành công đến API giả lập và trả về trạng thái ``Synced''. \\ \hline

\end{tabularx}
\end{table}
