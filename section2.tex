\section{Process Model and Planning}

Nhóm quyết định lựa chọn mô hình Scrum để phát triển hệ thống SERS.

Lý do lựa chọn:
\begin{itemize}
    \item Scrum phù hợp với nhóm 5 thành viên, giúp tăng cường giao tiếp và tính minh bạch thông qua các sự kiện Daily Scrum (dù là online hay offline).
    \item Thời gian phát triển ngắn (vài tuần) đòi hỏi khả năng phản hồi, thích ứng nhanh với các thay đổi. Việc chia nhỏ dự án thành các Sprint 1 tuần giúp nhóm kiểm soát tiến độ tốt hơn so với mô hình Waterfall.
    \item Hệ thống được định nghĩa dựa trên các User Stories (như đã phân tích ở phần Requirements), tương thích với Product Backlog của Scrum.
\end{itemize}

% \subsection{Project Planning}
Dự án kéo dài 2 tuần, được chia làm 2 Sprints.
\begin{itemize}
    \item \textbf{Sprint 1 (Tuần 1):} Tập trung vào các tính năng cốt lõi (Core Features) - MVP.
    \item \textbf{Sprint 2 (Tuần 2):} Hoàn thiện các tính năng nâng cao, Notification và Refactoring.
\end{itemize}

\subsection{Product Backlog \& Planning Poker Estimation}
Nhóm sử dụng kỹ thuật Planning Poker \cite{mgs_planning-poker_2025} để ước lượng độ phức tạp (Effort) cho từng User Story theo dãy số Fibonacci (0, 1, 2, 3, 5, 8, 13, 21).

\begin{table}[h!]
\centering
\begin{tabular}{|c|p{8cm}|c|c|}
\hline
\textbf{ID} & \textbf{User Story} & \textbf{Priority} & \textbf{Story Points} \\ \hline
US-01 & Sinh viên tạo và xem danh sách sự kiện (CRUD) & Must Have & 8 \\ \hline
US-02 & Gia sư nhận thông báo nhắc nhở (Notification) & Must Have & 8 \\ \hline
US-04 & Bệnh nhân tạo nhắc nhở lặp lại (Recurring Events) & Should Have & 5 \\ \hline
US-03 & Chia sẻ sự kiện qua URL (Sharing) & Could Have & 3 \\ \hline
US-05 & Đồng bộ Google Calendar & Won't Have & - \\ \hline
\multicolumn{3}{|r|}{\textbf{Total Points}} & \textbf{24} \\ \hline
\end{tabular}
\caption{Product Backlog với ước lượng điểm}
\end{table}

\subsection{Sprint Backlogs}

\subsubsection*{Sprint 1: Core Mechanics (Velocity dự kiến: 13 points)}
\textbf{Mục tiêu:} Người dùng có thể tạo, lưu trữ và xem danh sách sự kiện.
\begin{itemize}
    \item \textbf{US-01 (8 points):} Thiết kế Database (Mock), API thêm/xóa/sửa sự kiện, Giao diện danh sách.
    \item \textbf{Part of US-04 (5 points):} Xử lý logic lặp lại cơ bản (Backend logic).
\end{itemize}

\subsubsection*{Sprint 2: Notification \& Polish (Velocity dự kiến: 11 points)}
\textbf{Mục tiêu:} Hoàn thiện hệ thống nhắc nhở và tính năng chia sẻ.
\begin{itemize}
    \item \textbf{US-02 (8 points):} Cài đặt Background Service kiểm tra giờ và gửi thông báo (Console log/Popup).
    \item \textbf{US-03 (3 points):} Tạo trang xem sự kiện public (Read-only view).
    \item \textbf{Testing \& Documentation:} Viết test case và hoàn thiện báo cáo.
\end{itemize}

\subsection{Burndown Chart (Simulated)}
Biểu đồ Burndown \cite{projectmanager_burndown_2025} dự kiến cho thấy sự giảm dần của Story Points qua 14 ngày.
% Bạn có thể chèn hình ảnh biểu đồ tự vẽ vào đây
\begin{figure}[h!]
    \centering
    \includegraphics[width=0.8\textwidth]{image/burndown.png}
    \caption{Burndown Chart cho 2 Sprints}
    \label{fig:burndown}
\end{figure}